\prechapter{RESUMEN}


\section{Introducción}
El proyecto \textbf{LIA} se centra en la monitorización de invernaderos utilizando luz artificial. Su objetivo es proporcionar una plataforma que permita a los usuarios controlar y optimizar el uso de recursos en entornos agrícolas mediante dispositivos IoT.

\section{Backend}
El backend está construido sobre un marco de trabajo especializado para dispositivos IoT, lo que permite la creación y gestión de modelos de datos adaptables a diferentes dispositivos. Esto incluye:

\begin{itemize}
    \item Generación de nuevos modelos de datos para dispositivos.
    \item Relación entre dispositivos y sus respectivos modelos de datos.
\end{itemize}

\section{Funcionalidades}
La plataforma LIA ofrecerá las siguientes funcionalidades:

\begin{itemize}
    \item Graficar datos en tiempo real para facilitar el análisis.
    \item Monitorizar el estado y rendimiento de los invernaderos.
    \item Crear y gestionar modelos de datos personalizados según las necesidades de los usuarios.
\end{itemize}

\section{Tecnologías}
Las tecnologías utilizadas en el desarrollo del proyecto incluyen:

\begin{itemize}
    \item \textbf{Node.js} y \textbf{Express} para el desarrollo del servidor.
    \item \textbf{TypeScript} y \textbf{JavaScript} para la lógica del backend y frontend.
    \item \textbf{React} para la construcción de interfaces de usuario interactivas.
    \item \textbf{MongoDB} como base de datos NoSQL para almacenar datos de dispositivos y modelos.
\end{itemize}

\makeatletter
\textbf{Palabras Clave:} \@palabras
\makeatother
% \lipsum[2-4]