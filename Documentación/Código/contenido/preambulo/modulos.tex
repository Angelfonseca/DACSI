\prechapter{Modulos del sistema}

\section{Módulo de Usuarios}
\begin{table}[h!]
    \centering
    \begin{tabular}{|c|c|c|}
    \hline
    \rowcolor[HTML]{FCE8B2} 
    \textbf{Campo} & \textbf{Tipo de Dato} & \textbf{Descripción} \\ \hline
    ID & Entero & Identificador único del usuario. \\ \hline
    Nombre & Cadena de texto & Nombre completo del usuario. \\ \hline
    Nombre de Usuario & Cadena de texto & Nombre único de identificación del usuario. \\ \hline
    Correo Electrónico & Cadena de texto & Dirección de correo del usuario. \\ \hline
    Rol & Cadena de texto & Rol del usuario en el sistema (Admin, Usuario). \\ \hline
    Contraseña & Cadena de texto & Contraseña encriptada para acceder al sistema. \\ \hline
    Fecha de Creación & Fecha/Hora & Momento en que se creó el usuario. \\ \hline
    \end{tabular}
    \caption{Especificaciones del Módulo de Usuarios.}
\end{table}

\section{Módulo de Sensores}
\begin{table}[h!]
    \centering
    \begin{tabular}{|c|c|c|}
    \hline
    \rowcolor[HTML]{FCE8B2} 
    \textbf{Campo} & \textbf{Tipo de Dato} & \textbf{Descripción} \\ \hline
    ID & Entero & Identificador único del sensor. \\ \hline
    Módulo & Cadena de texto & Nombre del módulo al que pertenece el sensor. \\ \hline
    Sensores & Arreglo & Conjunto de sensores, cada uno con propiedades como: \\ \hline
    \hspace{5mm} Sensor & Cadena de texto & Tipo de sensor (Temperatura, Humedad, Luz). \\ \hline
    \hspace{5mm} Amp & Decimal & Valor de amperaje reportado por el sensor. \\ \hline
    \hspace{5mm} Volt & Decimal & Valor de voltaje reportado por el sensor. \\ \hline
    \hspace{5mm} Pot & Decimal & Valor de potencia reportado por el sensor. \\ \hline
    Fecha de Creación & Fecha/Hora & Momento en que se creó el registro del sensor. \\ \hline
    \end{tabular}
    \caption{Especificaciones del Módulo de Sensores.}
\end{table}

\section{Módulo de Graficación}
\begin{table}[h!]
    \centering
    \begin{tabular}{|c|c|c|}
    \hline
    \rowcolor[HTML]{FCE8B2} 
    \textbf{Campo} & \textbf{Tipo de Dato} & \textbf{Descripción} \\ \hline
    ID & Entero & Identificador único de la gráfica. \\ \hline
    Tipo de Gráfica & Cadena de texto & Tipo de visualización (Línea, Barras). \\ \hline
    Datos & JSON & Datos a graficar en formato JSON. \\ \hline
    Rango de Fechas & Fecha/Hora & Intervalo de tiempo para el que se generan los datos. \\ \hline
    Fecha de Creación & Fecha/Hora & Momento en que se creó el registro de la gráfica. \\ \hline
    \end{tabular}
    \caption{Especificaciones del Módulo de Graficación.}
\end{table}

\section{Módulo de Alertas}
\begin{table}[h!]
    \centering
    \begin{tabular}{|c|c|c|}
    \hline
    \rowcolor[HTML]{FCE8B2} 
    \textbf{Campo} & \textbf{Tipo de Dato} & \textbf{Descripción} \\ \hline
    ID & Entero & Identificador único de la alerta. \\ \hline
    Tipo de Alerta & Cadena de texto & Tipo de alerta (Crítica, Advertencia). \\ \hline
    Mensaje & Cadena de texto & Descripción de la alerta generada. \\ \hline
    Fecha de Generación & Fecha/Hora & Momento en que se generó la alerta. \\ \hline
    Estado & Cadena de texto & Estado de la alerta (Leída, No leída). \\ \hline
    \end{tabular}
    \caption{Especificaciones del Módulo de Alertas.}
\end{table}

