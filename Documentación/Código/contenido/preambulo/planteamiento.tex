\prechapter{PLANTEAMIENTO DEL PROBLEMA}

\section{Planteamiento del Problema}

El Tecnológico de Pabellón de Arteaga está llevando a cabo una investigación innovadora denominada \textbf{LIA}, cuyo objetivo es estudiar el crecimiento de plantas mediante el uso de luz artificial. Este proyecto busca optimizar las condiciones de cultivo y mejorar la eficiencia en el uso de recursos, lo que es fundamental en el contexto actual de la agricultura sostenible y la producción alimentaria.

Sin embargo, el proyecto LIA enfrenta una serie de desafíos en su fase de implementación. Actualmente, se utilizan dispositivos como la Raspberry Pi para la medición de variables del sistema, tales como la temperatura, la humedad y la intensidad de luz. Aunque estos dispositivos permiten la recolección de datos, la capacidad de análisis y gestión de dicha información es limitada. A medida que el proyecto crece, se hace necesario contar con un sistema robusto que facilite no solo la medición, sino también el almacenamiento, la monitorización en tiempo real y la visualización gráfica de los datos obtenidos.

El desarrollo de un \textbf{framework para dispositivos IoT} se presenta como la solución ideal para abordar estos problemas. Este framework permitirá:

\begin{itemize}
    \item \textbf{Almacenamiento de Datos}: Guardar de manera estructurada la información recolectada, lo que facilitará el acceso y la recuperación de datos históricos.
    \item \textbf{Monitorización en Tiempo Real}: Proporcionar a los investigadores herramientas que les permitan observar las variables del sistema al instante, mejorando la toma de decisiones y la respuesta ante variaciones.
    \item \textbf{Visualización Gráfica}: Ofrecer representaciones gráficas de los datos, lo que facilitará el análisis comparativo entre diferentes variables y condiciones experimentales.
\end{itemize}

Estos elementos son cruciales para el éxito del proyecto LIA, ya que no solo permitirán un control más efectivo del entorno de crecimiento, sino que también proporcionarán a los investigadores una plataforma que les permita realizar análisis más profundos y fundamentados. En consecuencia, el desarrollo de este framework será un paso determinante para la evolución del proyecto, permitiendo un enfoque más sistemático y científico en la investigación sobre el crecimiento de plantas con luz artificial.