\prechapter{ABSTRACT}
\large{El presente trabajo de tesis se centra en el desarrollo de una plataforma de monitorización para invernaderos de luz artificial, denominada \textbf{LIA}. Esta plataforma utiliza un backend basado en un marco de trabajo especializado para dispositivos IoT, lo que permite la creación y gestión de modelos de datos adaptables a diversos dispositivos de monitoreo. LIA ofrece funcionalidades para graficar y monitorizar en tiempo real los parámetros críticos de los invernaderos, así como para crear modelos de datos personalizados según las necesidades específicas de los usuarios. La implementación se lleva a cabo utilizando tecnologías modernas como Node.js, TypeScript, JavaScript, React y MongoDB. Este enfoque no solo optimiza el uso de recursos, sino que también mejora la toma de decisiones en la gestión de cultivos, contribuyendo así a la sostenibilidad y eficiencia en la agricultura de precisión.}


\makeatletter
\textbf{Keywords:} \@keywords
\makeatother