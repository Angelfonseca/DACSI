\prechapter{INTRODUCCIÓN}

\section{Introducción}

La agricultura de precisión ha revolucionado la forma en que se cultivan los productos agrícolas, optimizando el uso de recursos y maximizando la productividad. Uno de los desarrollos más prometedores en este ámbito es el uso de invernaderos de luz artificial, que permiten un control preciso sobre las condiciones de crecimiento, como la temperatura, la humedad y la iluminación. Sin embargo, la gestión eficiente de estos sistemas requiere herramientas avanzadas de monitorización y control, capaces de proporcionar datos en tiempo real y facilitar la toma de decisiones informadas.

El proyecto \textbf{LIA} se presenta como una solución innovadora para la monitorización de invernaderos de luz artificial. Su objetivo principal es desarrollar una plataforma que integre tecnologías de Internet de las Cosas (IoT) para la recolección y análisis de datos, permitiendo a los agricultores optimizar el funcionamiento de sus invernaderos. A través de la implementación de un backend robusto, se busca la capacidad de generar y gestionar modelos de datos que se adapten a las características de cada dispositivo de monitoreo, así como la capacidad de relacionar estos modelos con los dispositivos correspondientes.

\subsection{Objetivos del Proyecto}
Los objetivos específicos del proyecto LIA incluyen:

\begin{itemize}
    \item Desarrollar un sistema de monitorización que permita graficar en tiempo real los datos relevantes de los invernaderos.
    \item Crear un framework que facilite la gestión de modelos de datos personalizados para diferentes dispositivos IoT.
    \item Implementar una interfaz de usuario interactiva que permita a los agricultores acceder a información crítica sobre el estado de sus invernaderos.
    \item Proporcionar herramientas analíticas que ayuden en la toma de decisiones sobre el manejo de cultivos y recursos.
\end{itemize}

\subsection{Justificación}
La implementación de una plataforma como LIA es crucial en el contexto actual, donde la demanda de alimentos sigue aumentando debido al crecimiento poblacional y a los cambios en los patrones de consumo. La agricultura de luz artificial permite el cultivo durante todo el año y en diversas condiciones climáticas, pero su efectividad depende en gran medida de la capacidad de monitoreo y control que se tenga. Por lo tanto, LIA no solo contribuirá a mejorar la eficiencia y sostenibilidad de los invernaderos, sino que también permitirá a los agricultores adaptarse a las exigencias del mercado contemporáneo.

\subsection{Estructura del Documento}
El presente documento está estructurado de la siguiente manera: en la sección \textbf{2}, se abordarán las tecnologías empleadas en el desarrollo de la plataforma; la sección \textbf{3} describirá el diseño y la arquitectura del sistema; en la sección \textbf{4}, se presentarán los resultados obtenidos y se discutirán sus implicaciones; finalmente, la sección \textbf{5} concluirá con un resumen de los hallazgos y las futuras líneas de investigación.