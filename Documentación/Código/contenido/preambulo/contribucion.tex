\prechapter{CONTRIBUCIÓN AL CONOCIMIENTO}

\section{Contribución al Conocimiento}

El proyecto \textbf{LIA} tiene el potencial de realizar contribuciones significativas al campo de la agricultura y la investigación en tecnologías de iluminación artificial para el crecimiento de plantas. A continuación, se detallan las principales áreas en las que este proyecto contribuirá al conocimiento científico y práctico:

\subsection{Desarrollo de Modelos de Crecimiento}
El marco que se desarrollará como parte del proyecto LIA permitirá la creación de modelos de crecimiento más precisos y adaptativos para diversas especies de plantas bajo condiciones de luz artificial. Estos modelos no solo se basarán en datos empíricos recolectados a través de la plataforma, sino que también integrarán variables ambientales, lo que permitirá una comprensión más profunda de las interacciones entre estas variables y el crecimiento vegetal.

\subsection{Generación de Datos Estandarizados}
El proyecto facilitará la recolección de datos estandarizados y estructurados sobre el crecimiento de plantas bajo diferentes condiciones controladas. Esta recopilación sistemática de datos será valiosa para futuros estudios y permitirá la comparación entre investigaciones, promoviendo una base de conocimiento común que puede ser utilizada por otros investigadores en el área de la agricultura de precisión.

\subsection{Avances en Tecnología IoT}
El desarrollo del framework para dispositivos IoT no solo beneficiará al proyecto LIA, sino que también puede servir como modelo para futuras investigaciones en la aplicación de tecnologías IoT en la agricultura. Este enfoque permitirá una integración más fluida de dispositivos de monitoreo y control en otros sistemas agrícolas, fomentando el desarrollo de soluciones tecnológicas innovadoras en el sector.

\subsection{Educación y Capacitación}
El proyecto LIA también contribuirá al conocimiento a través de la educación y capacitación de estudiantes e investigadores en el uso de tecnologías de monitorización y análisis de datos. La experiencia adquirida por los participantes en el desarrollo y uso de la plataforma enriquecerá su formación y los preparará para futuros desafíos en el campo de la agricultura y la tecnología.

\subsection{Sostenibilidad Agrícola}
Al proporcionar datos y análisis que permitan optimizar el uso de recursos en la agricultura, el proyecto LIA contribuirá a la sostenibilidad en la producción agrícola. Los hallazgos derivados de la investigación pueden ayudar a establecer prácticas más eficientes que reduzcan el desperdicio y promuevan un uso responsable de los recursos, lo cual es fundamental para enfrentar los desafíos globales de seguridad alimentaria.

\subsection{Fomento de Investigaciones Futuras}
Finalmente, los resultados y datos generados por el proyecto LIA abrirán nuevas líneas de investigación en el campo de la agricultura y la tecnología de cultivo. Otros investigadores podrán utilizar estos datos como base para desarrollar estudios adicionales, contribuyendo así al avance del conocimiento en áreas relacionadas como la biotecnología, la genética de plantas y la agricultura sostenible.

\subsection{Proyecto Open Source}
Como parte de su compromiso con la difusión del conocimiento y la colaboración, el proyecto LIA será desarrollado como un proyecto open source. Esto permitirá que investigadores, desarrolladores y agricultores que requieran soluciones similares puedan acceder, modificar y contribuir a la plataforma, fomentando un ambiente de innovación y mejora continua. Además, el proyecto se exhibirá en el \textbf{COSS} (Centro de Observación y Sensores de Sostenibilidad) en Guadalajara en el año 2025, brindando visibilidad y oportunidades de colaboración con otros actores en el ámbito de la tecnología agrícola.

En resumen, el proyecto LIA no solo busca resolver problemas inmediatos relacionados con el crecimiento de plantas mediante luz artificial, sino que también se propone contribuir de manera significativa al acervo de conocimientos en el ámbito agrícola y tecnológico. Su impacto se extenderá más allá de los resultados inmediatos, fomentando un enfoque más científico y sistemático en la investigación y desarrollo agrícola.
