\prechapter{JUSTIFICACIÓN}

\section{Justificación}

El desarrollo del proyecto \textbf{LIA} se fundamenta en la necesidad urgente de optimizar la producción agrícola en un mundo que enfrenta desafíos significativos en términos de seguridad alimentaria y sostenibilidad. A medida que la población global sigue creciendo, también lo hace la demanda de alimentos, lo que requiere que los métodos tradicionales de cultivo sean revisados y mejorados. La agricultura de precisión y el uso de tecnologías avanzadas, como la luz artificial y los sistemas de monitorización, se presentan como soluciones viables para abordar estos retos.

La justificación del proyecto LIA se puede desglosar en los siguientes puntos:

\subsection{Innovación en la Agricultura}
El uso de luz artificial para el crecimiento de plantas es un enfoque innovador que permite el cultivo en condiciones controladas, independientemente de factores climáticos externos. Esto no solo aumenta la eficiencia de la producción, sino que también permite el cultivo de variedades de plantas que de otro modo no prosperarían en ciertas regiones. La implementación de un sistema de monitorización y análisis de datos a través de un framework para dispositivos IoT permitirá maximizar los beneficios de esta tecnología.

\subsection{Mejora de la Toma de Decisiones}
El marco propuesto facilitará la recopilación y análisis de datos en tiempo real, lo que permitirá a los investigadores tomar decisiones informadas sobre el manejo de las variables del sistema. La capacidad de monitorizar condiciones como la temperatura, humedad e intensidad de luz, y visualizar estos datos en forma gráfica, contribuirá a una mejor comprensión de los efectos de diferentes parámetros en el crecimiento de las plantas. Esto es fundamental para la investigación científica y la mejora de los métodos de cultivo.

\subsection{Accesibilidad a la Información}
El desarrollo del framework no solo beneficiará a los investigadores del Tecnológico de Pabellón de Arteaga, sino que también tiene el potencial de ser una herramienta valiosa para otros institutos de investigación y agricultores. Al proporcionar acceso a datos estructurados y herramientas de análisis, el proyecto LIA contribuirá a la democratización del conocimiento en el ámbito de la agricultura de luz artificial.

\subsection{Sostenibilidad y Eficiencia}
La capacidad de monitorizar y controlar con precisión las variables ambientales en los invernaderos permitirá un uso más eficiente de los recursos, como el agua y la energía. Esto es especialmente relevante en un contexto global donde la sostenibilidad es una prioridad. La optimización de estos recursos no solo ayudará a reducir costos, sino que también contribuirá a la protección del medio ambiente.

\subsection{Contribución al Conocimiento Científico}
Finalmente, el proyecto LIA ofrecerá un marco sólido para la investigación y el desarrollo de nuevas tecnologías en el ámbito de la agricultura. La recopilación y análisis de datos proporcionarán una base empírica que podrá ser utilizada en estudios futuros, lo que contribuirá a un mejor entendimiento de las dinámicas del crecimiento de las plantas bajo luz artificial y abrirá nuevas líneas de investigación.

En resumen, la justificación del proyecto LIA radica en su capacidad para innovar en la agricultura, mejorar la toma de decisiones, facilitar el acceso a la información, promover la sostenibilidad y contribuir al conocimiento científico. Estas razones destacan la relevancia y la urgencia de llevar a cabo este proyecto en el contexto actual.