\prechapter{Entrevista}
\begin{itemize} 
    \item \textbf{¿Qué es lo que se quiere?} \ El propósito es desarrollar un sistema que permita monitorizar las variables de salida de los sensores del invernadero, que incluyen voltaje, potencia y amperaje. Estas son las principales variables que se medirán, y cada sensor se diferenciará por su nombre. Además, se busca la capacidad de graficar los datos según diferentes fechas o comparar los valores de distintos sensores en fechas específicas.

    \item \textbf{¿Quién será el usuario principal del sistema y cuál es su nivel de experiencia técnica?} \ Solo habrá un tipo de usuario. Se espera que los usuarios tengan una interacción sencilla con el sistema y que este sea fácil de entender, independientemente de su nivel de experiencia técnica.

    \item \textbf{¿Qué información necesitan los usuarios para tomar decisiones clave en el día a día?} \ La información que los usuarios necesitarán dependerá de las variables medidas (voltaje, potencia y amperaje), para poder analizar el estado del invernadero y tomar decisiones informadas basadas en los datos recolectados de los sensores.

    \item \textbf{¿Cómo se espera que los usuarios interactúen con el sistema?} \ Se espera que la interacción con el sistema sea simple y amigable, facilitando el acceso a la información necesaria sin complicaciones técnicas.

    \item \textbf{¿Qué tan crucial es el monitoreo en tiempo real para la operación del invernadero?} \ La monitorización en tiempo real es prioritaria para revisar el estado actual de los diferentes dispositivos del invernadero y poder tomar acciones oportunas si es necesario.

    \item \textbf{¿Cómo se deben gestionar y presentar las alertas críticas?} \ Las alertas críticas aún no están especificadas, pero se espera que el sistema tenga la capacidad de generarlas en caso de detectar anomalías importantes en los datos.

    \item \textbf{¿Se espera que el sistema no solo monitoree, sino que también automatice acciones en el invernadero?} \ No se espera que el sistema automatice acciones en el invernadero; su objetivo principal es el monitoreo y la visualización de datos.

    \item \textbf{¿Qué otros sistemas o herramientas están actualmente en uso y cómo se integraría el nuevo sistema?} \ Actualmente, no tienen ningún otro sistema en uso, por lo que el nuevo sistema será implementado desde cero sin necesidad de integración con herramientas existentes.

    \item \textbf{¿Hay alguna preferencia en cuanto a la tecnología o herramientas específicas para la implementación del sistema?} \ No existe preferencia hacia ninguna herramienta o tecnología en específico para la implementación del sistema.

    \item \textbf{¿Cuál es el presupuesto y el cronograma para la implementación del sistema?} \ [Aquí agregar la respuesta correspondiente] 
\end{itemize}