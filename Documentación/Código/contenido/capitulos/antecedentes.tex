\chapter{ANTECEDENTES}

\section{Antecedentes}

\subsection{Agricultura y Tecnología}
El avance de la tecnología ha tenido un impacto significativo en múltiples sectores, y la agricultura no ha sido la excepción. A lo largo de las últimas décadas, el concepto de \textbf{agricultura de precisión} ha ganado tracción, al integrarse con tecnologías emergentes como el Internet de las Cosas (IoT), sensores inteligentes y análisis de grandes volúmenes de datos (big data). Estas innovaciones han permitido mejorar la eficiencia en la producción agrícola, reducir el uso de recursos naturales y optimizar las condiciones para el crecimiento de las plantas. Sin embargo, uno de los grandes retos a los que se enfrenta la agricultura es la sostenibilidad en áreas de cultivo con acceso limitado a luz natural. Aquí es donde la \textbf{iluminación artificial} ha comenzado a jugar un papel clave.

\subsection{El Uso de la Iluminación Artificial en la Agricultura}
La luz es un factor crucial para la fotosíntesis, el proceso que permite a las plantas transformar la energía lumínica en compuestos químicos necesarios para su crecimiento. En muchos entornos controlados, como los invernaderos, la luz natural no siempre es suficiente para asegurar un crecimiento óptimo, lo que ha llevado al uso de \textbf{luz artificial} como complemento o sustituto. A medida que se han desarrollado nuevas tecnologías de iluminación, como las luces LED ajustables en longitud de onda, los invernaderos han podido recrear condiciones óptimas de luz para distintas especies vegetales. La capacidad de manipular el espectro lumínico ha permitido optimizar el crecimiento en condiciones que anteriormente eran inviables, promoviendo así un crecimiento más controlado y eficiente.

\subsection{Investigaciones Previas}
En los últimos años, se han llevado a cabo diversos estudios que exploran los efectos de la luz artificial en el crecimiento de plantas. Investigaciones como las de \textit{Smith et al.} (2020) y \textit{González et al.} (2019) han demostrado que ajustar los niveles de luz y la longitud de onda puede impactar directamente en el ciclo de crecimiento de varias especies. Sin embargo, la mayoría de estas investigaciones se han centrado en entornos experimentales y no en la aplicación a gran escala con tecnologías de IoT para la monitorización en tiempo real. Además, aunque se han logrado avances en sistemas de monitorización, aún existe una brecha en la integración de dispositivos IoT que permitan analizar y gestionar eficientemente las variables del entorno que influyen en el crecimiento vegetal, tales como la temperatura, la humedad y la cantidad de luz recibida.

\subsection{Internet de las Cosas (IoT) en la Agricultura}
El \textbf{Internet de las Cosas} (IoT, por sus siglas en inglés) ha emergido como una de las tecnologías clave para la transformación digital de la agricultura. IoT se refiere a la interconexión de dispositivos y sensores inteligentes a través de Internet, lo que permite la captura de datos en tiempo real y la automatización de procesos. En el contexto agrícola, los dispositivos IoT pueden monitorizar múltiples variables, como la temperatura del suelo, la humedad, los niveles de luz, y otros factores críticos para el crecimiento de las plantas. La capacidad de tomar decisiones informadas basadas en datos recolectados en tiempo real no solo mejora la eficiencia, sino que también reduce los costos y minimiza el impacto ambiental.

Varios proyectos a nivel mundial han utilizado IoT para optimizar la producción agrícola. Por ejemplo, en \textit{Xiang et al.} (2021) se desarrolló un sistema de sensores integrados para invernaderos, lo que permitió ajustar automáticamente las condiciones ambientales en función de los datos recolectados. Este enfoque de monitoreo inteligente es cada vez más relevante para maximizar la productividad en espacios controlados como los invernaderos.

\subsection{Proyecto LIA y Tecnológico de Pabellón de Arteaga}
El \textbf{Tecnológico de Pabellón de Arteaga} ha lanzado una iniciativa llamada \textbf{LIA (Luz Artificial para la Agricultura)}, cuyo objetivo es investigar y desarrollar tecnologías que mejoren el crecimiento de las plantas utilizando iluminación artificial. Esta iniciativa surge ante la necesidad de crear entornos controlados donde la luz natural no sea el principal recurso para la fotosíntesis. Como parte de este proyecto, se busca implementar un sistema de monitorización inteligente que utilice una combinación de dispositivos IoT, como \textbf{Raspberry Pi}, para medir variables ambientales y controlar el sistema de iluminación. El reto, sin embargo, radica en la integración de estos dispositivos en una plataforma robusta que permita analizar, graficar y gestionar los datos generados por los sensores.

La integración de iluminación artificial en la agricultura ha mostrado grandes beneficios, especialmente en entornos donde la luz natural es insuficiente para las necesidades de las plantas. ("Estudios recientes han demostrado que la manipulación del espectro de luz puede afectar el crecimiento de las plantas") \cite{gonzalez2019, zhang2021}. ("Las luces LED, con la capacidad de ajustar su espectro lumínico, ofrecen nuevas oportunidades para la agricultura moderna") \cite{zhang2021}.

\subsection{Framework para Dispositivos IoT}
Como solución a esta problemática, se plantea el desarrollo de un \textbf{framework para dispositivos IoT}, diseñado específicamente para monitorear las variables ambientales en los invernaderos. Este framework permitirá no solo la captura y almacenamiento de datos en tiempo real, sino también la creación de modelos personalizados para los dispositivos que monitoricen los parámetros de crecimiento de las plantas. Gracias a su flexibilidad, los investigadores podrán adaptar el sistema a las necesidades específicas de cada especie vegetal, facilitando la toma de decisiones en el manejo de los cultivos.

El framework está diseñado para ser escalable y abierto, por lo que podrá ser utilizado por otros proyectos que requieran soluciones similares. Su arquitectura se basa en tecnologías modernas como \textbf{Node.js}, \textbf{TypeScript}, \textbf{MongoDB}, y \textbf{React}, asegurando un alto rendimiento y una experiencia de usuario interactiva. Además, se espera que este framework pueda ser implementado en otros entornos agrícolas o industriales donde la monitorización en tiempo real de variables críticas sea indispensable.


\subsection{Conclusión}
A través de estos antecedentes, se ha establecido el contexto necesario para comprender la relevancia del proyecto LIA. La integración de la tecnología IoT en la agricultura no solo aborda problemas relacionados con la eficiencia y la productividad, sino que también contribuye a la sostenibilidad agrícola en un mundo donde el acceso a recursos naturales es cada vez más limitado. El proyecto LIA se propone como una solución tecnológica innovadora que permitirá a los investigadores y agricultores monitorear en tiempo real las condiciones de crecimiento, contribuyendo al avance de la agricultura moderna y sostenible.
("Por otro lado, los frameworks de IoT han evolucionado para ofrecer soluciones modulares y escalables, permitiendo la creación de modelos personalizados de monitoreo y control para distintas aplicaciones, incluidos los sistemas agrícolas") \cite{lopez2020, fernandez2021}
