\documentclass[12pt]{report}
\usepackage[a4paper]{geometry} % Configuración de márgenes y dimensiones de página

\usepackage[utf8]{inputenc} % Codificación de caracteres
\usepackage[spanish]{babel} % Configuración de idioma español
\usepackage{amssymb} % Símbolos matemáticos adicionales
\usepackage{subfigure} % Creación de subfiguras dentro de una figura
\usepackage{newunicodechar} % Soporte para caracteres Unicode
\usepackage{graphicx} % Inclusión de imágenes
\usepackage{amsmath} % Ampliación de funcionalidades matemáticas
\usepackage{xspace} % Control del espacio después de comandos
\usepackage{booktabs} % Mejora la apariencia de las tablas
\usepackage{tabularx}
\usepackage[table,xcdraw]{xcolor} % Colores personalizados para tablas
\usepackage{url} % Inclusión y formato de enlaces URL
\usepackage{lineno} % Numeración de líneas
\usepackage{enumitem} % Control de listas y enumeraciones
\usepackage{soul} % Resaltado de texto con subrayado
\usepackage[numbers]{natbib} % Citas bibliográficas y bibliografía
\usepackage[colorlinks=true, bookmarks=false, citecolor=blue, urlcolor=blue, linkcolor=blue, linktoc=page]{hyperref} % Creación de enlaces y referencias
\usepackage{titlesec} % Control del formato de títulos de sección
\usepackage{fancyhdr} % Personalización de encabezados y pies de página
\usepackage{datetime} % Manipulación de fechas y tiempos
% Modificar márgenes
\geometry{left=2.5cm, right=2.5cm, top=2.0cm, bottom=2.0cm}
% Espaciado entre líneas
\linespread{1.5} 
% Cambiar nombre a tablas
\addto\captionsspanish{\renewcommand{\tablename}{Tabla}}
% Cambiar nombre a lista de tablas
\addto\captionsspanish{\renewcommand{\listtablename}{Índice de tablas}}
% Modificar el estilo de los capítulos
\titleformat{\chapter}[display]
{\color{black}\fontfamily{pcr}\LARGE\scshape\raggedleft}
{\chaptertitlename\ \color{red}\fontsize{20}{00}\mdseries\thechapter} {-10pt} {\color{black}\rmfamily\huge}
%\titlespacing*{\chapter}{0pt}{0pt}{50pt}
% Diseño del encabezado y pie de página
\pagestyle{fancy}
\setlength{\headheight}{14.49998pt} % Ajustar la altura de la cabecera
\addtolength{\topmargin}{-2.49998pt} % Compensar ajustando el margen superior
\fancyhf{}
\fancyhead[L]{\nouppercase{\leftmark}}
\fancyhead[R]{\nouppercase{\rightmark}}
\fancyfoot[C]{\thepage}
\renewcommand{\headrulewidth}{0.2pt}
% Quitar espacio vertical en listas
\setlist[itemize]{noitemsep, topsep=0pt}
% Numerar las secciones
\setcounter{secnumdepth}{4} % Establece la profundidad de numeración de secciones
\titleformat{\paragraph}{\normalfont\normalsize\bfseries}{\theparagraph}{1em}{} % Define el formato de las subsubsubsecciones
% Modificar la tabla de contenido
\setcounter{tocdepth}{4}
\addto\captionsspanish{\renewcommand{\contentsname}{Contenido}}


%%%%%%%%%%%%%%%%%%%%%%%%%%% AQUI INICIA EL DOCUMENTO %%%%%%%
\begin{document}
\pagenumbering{Roman}
%%%%%%%%%%%%%%%%%%%%%%%%%%% PORTADA %%%%%%%%%%%%%%%%%%%%%%%%
\begin{titlepage}
\newgeometry{left=2.5cm,right=2.5cm,top=2.5cm,bottom=2.5cm}
\begin{minipage}[c]{0.15\textwidth}
  \centering
  \includegraphics[width=\textwidth]{01_Figuras/00_TECNM.png}
\end{minipage}
\hfill % Espacio horizontal entre las cajas
\begin{minipage}[c]{0.8\textwidth}
    \centering
        \textsc{\large Tecnológico Nacional de México campus: Pabellón de Arteaga}\\[0.3cm]
        \textsc{\large Tecnologías de la información y comunicaciones}\\[0.3cm]
        \hrule height1pt
        \vspace{0.1cm}
        \hrule height2pt
        \vspace{0.3cm}
\end{minipage}

\begin{minipage}[c]{0.15\textwidth}
    \vspace{0.5cm}
    \hspace{0.5cm}
    \vrule width1pt height15cm 
    \hspace{0.05cm}
    \vrule width2pt height15cm
\end{minipage}
\hfill % Espacio horizontal entre las cajas
\begin{minipage}[c]{0.8\textwidth}
    \centering
        \textbf{\large Monitorización de invernaderos }\\[1cm]
        \text{\large Por}\\[1cm]
        \text{\Large Ángel Isaac Fonseca Gómez}\\[1cm]
        \textit{\large Primer avance de tesis presentado al Departamento de Ingeniería como parte del proceso para obtener el grado de:}\\[1cm]
        \textbf{\Large Ingeniero en Tecnologías de la información y comunicación.}\\[1cm]
        \text{\large Dirigido por:}\\[1cm]
        \text{\Large PhD. Ernesto Olvera}\\[1cm]
\end{minipage}

\begin{minipage}[c]{0.15\textwidth}
  \centering
\end{minipage}
\hfill % Espacio horizontal entre las cajas
\begin{minipage}[c]{0.8\textwidth}
    \centering
        \mdseries{\large Pabellón de Arteaga, Aguascalientes, \the\year}
\end{minipage}

\end{titlepage}
\restoregeometry
%%%%%%%%%%%%%%%%%%%%%%%%%%% RESUMEN %%%%%%%%%%%%%%%%%%%%%%%%%%%%%
\thispagestyle{empty}
\newgeometry{left=2.5cm,right=2.5cm,top=2.5cm,bottom=2.5cm}
\addcontentsline{toc}{chapter}{Resumen}
\begin{center}
\textsc{\Huge Resumen}\\[0.5cm]
\textbf{\large Monitorización de invernaderos}\\[0.25cm]
\text{\large Por: Ángel Isaac Fonseca Gómez}\\[0.5cm]
\end{center}



\begin{flushright}
    \textit{\large Dirigido por:}\\
    \textit{\large Vicente Rico Ramírez, PhD.}
\end{flushright}


\restoregeometry
%%%%%%%%%%%%%%%%%%%%%%%%%%% TABLA DE CONTENIDO %%%%%%%%%%%%%%%%%
\tableofcontents

\newpage
\pagenumbering{arabic}
\chapter{Introducción}
\chapter{Objetivo}
\large{Implementar un sistema web de monitorización inteligente, con la opción de graficar la información para mejorar la comprensión de los datos brindados por los sensores.}\\[1cm]
\large{Con los siguientes objetivos especificos: }
\begin{itemize}
    \item Proveer fácil comprensión de los datos.
    \item Permitir revisión histórica de datos.
    \item Facilitar la fácil monitorización en tiempo real con los sensores.
    \item Acceso remoto al sistema.
    \item Generar modularización para facilitar el crecimiento.
\end{itemize}
\chapter{Resumen entrevista}
\begin{itemize} \item \textbf{¿Cuál es el propósito final del sistema?} \ [Aquí agregar la respuesta correspondiente] \item \textbf{¿Quién será el usuario principal del sistema y cuál es su nivel de experiencia técnica?} \ [Aquí agregar la respuesta correspondiente] \item \textbf{¿Qué información necesitan los usuarios para tomar decisiones clave en el día a día?} \ [Aquí agregar la respuesta correspondiente] \item \textbf{¿Cómo se espera que los usuarios interactúen con el sistema?} \ [Aquí agregar la respuesta correspondiente] \item \textbf{¿Qué tan crucial es el monitoreo en tiempo real para la operación del invernadero?} \ [Aquí agregar la respuesta correspondiente] \item \textbf{¿Cómo se deben gestionar y presentar las alertas críticas?} \ [Aquí agregar la respuesta correspondiente] \item \textbf{¿Se espera que el sistema no solo monitoree, sino que también automatice acciones en el invernadero?} \ [Aquí agregar la respuesta correspondiente] \item \textbf{¿Cuál es la frecuencia de recolección de datos necesaria para garantizar una toma de decisiones eficiente?} \ [Aquí agregar la respuesta correspondiente] \item \textbf{¿Qué desafíos actuales enfrentan con la infraestructura existente?} \ [Aquí agregar la respuesta correspondiente] \item \textbf{¿Cuántos usuarios deben acceder simultáneamente al sistema, y desde qué dispositivos?} \ [Aquí agregar la respuesta correspondiente] \item \textbf{¿Cómo se debería gestionar el mantenimiento y el soporte del sistema a lo largo del tiempo?} \ [Aquí agregar la respuesta correspondiente] \item \textbf{¿Existen expectativas de escalabilidad?} \ [Aquí agregar la respuesta correspondiente] \item \textbf{¿Qué tipo de datos históricos se consideran críticos, y durante cuánto tiempo deben almacenarse?} \ [Aquí agregar la respuesta correspondiente] \item \textbf{¿Cuáles son las situaciones de emergencia o fallos más comunes que deben ser gestionados automáticamente por el sistema?} \ [Aquí agregar la respuesta correspondiente] \item \textbf{¿Qué tipo de visualizaciones o reportes son necesarios para facilitar la comprensión de los datos?} \ [Aquí agregar la respuesta correspondiente] \end{itemize}

\chapter{Requerimientos}

Este capítulo describe los requerimientos que el sistema de monitorización de invernaderos debe cumplir para garantizar su correcta operación. Los requerimientos se dividen en dos secciones: funcionales y no funcionales, destacando aquellos que pueden ser medidos de manera objetiva.

\section{Requerimientos Funcionales}


\section{Requerimientos No Funcionales}

\begin{table}
\centering
\begin{tabularx}{\textwidth}{|c|X|c|}
\hline
\textbf{Nombre} & \textbf{Descripción} & \textbf{Versión} \\ \hline
\textbf{RNF-01: Tiempo de respuesta del sistema} & El tiempo de respuesta para cargar cualquier vista no debe exceder los 2 segundos bajo condiciones normales de carga con hasta 50 usuarios simultáneos. & 1.0 \\ \hline
\textbf{RNF-02: Escalabilidad del sistema} & El sistema debe escalar horizontalmente, soportando hasta 100 invernaderos adicionales sin que el tiempo de respuesta supere los 3 segundos. & 1.0 \\ \hline
\textbf{RNF-03: Disponibilidad del sistema} & El sistema debe estar disponible el 99.9\% del tiempo, permitiendo un máximo de 8 horas de inactividad al año. & 1.0 \\ \hline
\textbf{RNF-04: Seguridad de la información} & Toda la información transmitida deberá estar cifrada mediante SSL/TLS para proteger la confidencialidad de los datos. & 1.0 \\ \hline
\textbf{RNF-05: Autenticación y control de acceso} & Solo administradores tendrán acceso. Se implementará autenticación multifactor (MFA), completándose en un tiempo máximo de 5 segundos. & 1.0 \\ \hline
\end{tabularx}
\caption{Requerimientos no funcionales}
\end{table}




\newpage
\bibliographystyle{elsarticle-num-names}4

\bibliography{03_Bibliografias.bib}
\end{document}
